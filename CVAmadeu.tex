%------------------------------------
% Dario Taraborelli
% Typesetting your academic CV in LaTeX
%
% URL: http://nitens.org/taraborelli/cvtex
% DISCLAIMER: This template is provided for free and without any guarantee 
% that it will correctly compile on your system if you have a non-standard  
% configuration.
% Some rights reserved: http://creativecommons.org/licenses/by-sa/3.0/
%------------------------------------

%!TEX TS-program = xelatex
%!TEX encoding = UTF-8 Unicode

\documentclass[10pt, a4paper]{article}
\usepackage{fontspec} 

% DOCUMENT LAYOUT
\usepackage{geometry} 
\geometry{a4paper, textwidth=5.5in, textheight=8.5in, marginparsep=7pt, marginparwidth=.6in}
\setlength\parindent{0in}

% FONTS
\usepackage[usenames,dvipsnames]{xcolor}
\usepackage{xunicode}
\usepackage{xltxtra}
\defaultfontfeatures{Mapping=tex-text}
%\setromanfont [Ligatures={Common}, Numbers={OldStyle}, Variant=01]{Linux Libertine O}
%\setmonofont[Scale=0.8]{Monaco}
%%% modified by Karol Kozioł for ShareLaTeX use
\setmainfont[
  Ligatures={Common}, Numbers={OldStyle}, Variant=01,
  BoldFont=LinLibertine_RB.otf,
  ItalicFont=LinLibertine_RI.otf,
  BoldItalicFont=LinLibertine_RBI.otf
]{LinLibertine_R.otf}
\setmonofont[Scale=0.8]{DejaVuSansMono.ttf}

% ---- CUSTOM COMMANDS
\chardef\&="E050
\newcommand{\html}[1]{\href{#1}{\scriptsize\textsc{[html]}}}
\newcommand{\pdf}[1]{\href{#1}{\scriptsize\textsc{[pdf]}}}
\newcommand{\doi}[1]{\href{#1}{\scriptsize\textsc{[doi]}}}
% ---- MARGIN YEARS
\usepackage{marginnote}
\newcommand{\amper{}}{\chardef\amper="E0BD }
\newcommand{\years}[1]{\marginnote{\scriptsize #1}}
\renewcommand*{\raggedleftmarginnote}{}
\setlength{\marginparsep}{7pt}
\reversemarginpar

% HEADINGS
\usepackage{sectsty} 
\usepackage[normalem]{ulem} 
\sectionfont{\mdseries\upshape\Large}
\subsectionfont{\mdseries\scshape\normalsize} 
\subsubsectionfont{\mdseries\upshape\large} 

% PDF SETUP
% ---- FILL IN HERE THE DOC TITLE AND AUTHOR
\usepackage[%dvipdfm, 
bookmarks, colorlinks, breaklinks, 
% ---- FILL IN HERE THE TITLE AND AUTHOR
pdftitle={Rodrigo R Amadeu - CV},
	pdfauthor={My name},
	pdfproducer={http://nitens.org/taraborelli/cvtex}
]{hyperref}  
\hypersetup{linkcolor=blue,citecolor=blue,filecolor=black,urlcolor=MidnightBlue} 

% DOCUMENT
\begin{document}
{\LARGE Rodrigo Rampazo Amadeu}\\[1cm]
 Fifield Hall\\
 2550  Hull Rd\\
Gainesville, FL, U.S.A.\\[.2cm]
email: \href{mailto:rramadeu@ufl.edu}{rramadeu@ufl.edu}\\
\textsc{url}: \href{https://rramadeu.github.io/}{https://rramadeu.github.io/}\\ 
Nationality:  Brazilian\\
\hrule
\section*{Current position}
\emph{Ph.D. Student \& Graduate Research Assistant}, \href{https://www.blueberrybreeding.com}{Blueberry Breeding \& Genomics Lab}\\
PI: Dr. Patricio Munoz\\
Horticultural Sciences Department, University of Florida, USA

%%\hrule
\section*{Areas of specialization}
 Plant Genetics \& Breeding • Statistical-Genetics

%%\hrule
%\section*{Appointments held}
%\noindent
%\years{1903-1908}Swiss Patent Office, Bern\\
%\years{1908-1911}University of Bern\\
%\years{1911-1912}University of Zürich\\
%\years{1912-1914}Charles University of Prague\\
%\years{1914-1932}Prussian Academy of Sciences, Berlin\\
%\years{1920-1930}University of Leiden\\
%\years{1932-1955}Institute for Advanced Study, Princeton

%\hrule
\section*{Education}
\noindent
\years{2018}\textsc{M.S.} in Plant Genetics and Breeding, \href{http://statgen.esalq.usp.br}{Statistical-Genetics Lab}, University of São Paulo, Brazil\\
\years{2016}\textsc{B.Eng.} in Agriculture, University of São Paulo, Brazil\\
\years{2016}\textsc{B.Edu.} in Agricultural Sciences, University of São Paulo, Brazil\\

%\hrule
%\section*{Grants, honors \& awards}
\section*{Awards \& Scholarships}
\noindent
\years{2019}Poster Competition Plant Science Symposium, University of Florida - 1\textsuperscript{st} Place Winner\\
\years{2016}Prof Friedrich Gustav Brieger Prize - Best graduating student of Department of Genetics\\
\years{2013}Scholarship, Science without Borders - CAPES - 1yr at University of Florida\\
\years{2012}Scholarship, Scientific Initiation - PIBIC/CNPq - 1yr\\
\years{2011}Scholarship, Scientific Initiation - Santander - 1yr\\

\section*{Journal Articles}

%\subsection*{Journal articles}
\noindent
\years{2019}de Bem Oliveira, \emph{et al}, "Genomic prediction of autotetraploids; influence of relationship matrices, allele Dosage, and continuous genotyping calls in phenotype prediction",  \emph{G3}, 3, \href{https://doi.org/10.1534/g3.119.400059}{link}\\
\years{2018}Conson, Taniguti \& Amadeu, \emph{et al}, "High-resolution genetic map and QTL analysis of growth-related traits of \emph{Hevea brasiliensis}",  \emph{Front. Plant Sci.}, 9(1255), \href{https://doi.org/10.3389/fpls.2018.01255}{link}\\
\years{2018}Ferreira \emph{et al}, "Metabolite profiles of sugarcane culm reveal the relationship among metabolism and axillary bud outgrowth in genetically related sugarcane commercial cultivars",  \emph{Front. Plant Sci.}, 9(857), \href{https://doi.org/10.3389/fpls.2018.00857}{link}\\
\years{2018}Cellon \emph{et al}, "Estimation of genetic parameters and prediction of breeding values in an autotetraploid blueberry breeding population with extensive pedigree data",  \emph{Euphytica}, 214(87), \href{https://doi.org/10.1007/s10681-018-2165-8}{link}\\
\years{2016}Amadeu \emph{et al}, "AGHmatrix: R package to construct relationship matrices for autotetraploid and diploid Species, a blueberry example",  \emph{The Plant Genome}, 9(3), \href{http://doi.org/10.3835/plantgenome2016.01.0009}{link}\\

\section*{Softwares}
\years{AGHmatrix}author, software to compute relationship matrices for diploid and autopolyploid species, \href{https://cran.r-project.org/package=AGHmatrix}{link}\\
\years{onemap}contributor, software for constructing genetic maps in experimental crosses, \href{https://cran.r-project.org/package=onemap}{link}\\
\years{onemap2pop}author, onemap extension for constructing multi-family genetic maps in outcrossing species, \href{https://www.github.com/augusto-garcia/onemap2pop}{link}\\
\years{fullsibQTL}co-author, software for QTL mapping in outcrossing species using composite interval mapping, \href{https://www.github.com/augusto-garcia/fullsibQTL}{link}\\
%\hrule

\section*{Skills}
\years{programming}R: advanced (package development, tidyverse, shiny/plotly app, parallelization)\\
\years{programming}shell script, ASReml, \LaTeX \\
\years{statistics}experimental analysis of genetic \& agricultural data\\
\years{language}english \& portuguese

\section*{Teaching}
\years{2010-...}several R programming courses \href{https://rramadeu.github.io/year-archive/}{link}\\
\years{2018}TA of Field Plot Techniques, grad, University of Florida\\
\years{2015}TA of Calculus I, undergrad, University of Sao Paulo\\
\years{2012}TA of Calculus I, undergrad, University of Sao Paulo\\
\years{2011-2015}Algebra instructor in a college prep school\\

%\vspace{1cm}
\vfill{}
%\hrulefill

\begin{center}
{\scriptsize  Last updated: \today\- •\- 
% ---- PLEASE LEAVE THIS BACKLINK FOR ATTRIBUTION AS PER CC-LICENSE
Typeset in \href{http://nitens.org/taraborelli/cvtex}{
%\fontspec{Times New Roman}
\XeTeX }\\
% ---- FILL IN THE FULL URL TO YOUR CV HERE
\href{https://github.com/rramadeu/cv/raw/master/CVAmadeu.pdf}{https://github.com/rramadeu/cv/raw/master/CVAmadeu.pdf}}
\end{center}

\end{document}
